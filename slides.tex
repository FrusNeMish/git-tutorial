\documentclass[pdf,russian]{beamer}

\usepackage[T2A]{fontenc}
\usepackage[utf8]{inputenc}
\usepackage[russian]{babel}
\usepackage{booktabs}
\usepackage{minted}
\usepackage{tikz}

\usetikzlibrary{trees}

\selectlanguage{russian}

\mode<presentation>{
    \usetheme{Frankfurt}
    \useoutertheme{infolines}
}

%% preamble
\title{Введение в git}
\author{Артем Оганджанян}
\institute{CSC}
\date{27 октября 2017 г.}

\begin{document}

\AtBeginSection[]
{
    \begin{frame}
        \frametitle{Оглавление}
        \tableofcontents
        [
            currentsection,
            subsectionstyle=show/show/hide
        ]
    \end{frame}
}

\begin{frame}
    \titlepage
\end{frame}

\section{Мотивация}
\subsection{История изменений}

\begin{frame}[fragile]
    \frametitle{Потеря кода}
    \pause
    Напишу-ка я тетрис.
    \begin{block}{}
        \begin{minted}{bash}
$ vim tetris.cpp
$ g++ tetris.cpp
$ ./a.out
        \end{minted}
    \end{block}
    \pause
    Добавлю разные уровни сложности...
    \begin{block}{}
        \begin{minted}{bash}
$ vim tetris.cpp
$ g++ tetris.cpp
$ ./a.out
        \end{minted}
    \end{block}
    \pause
    \only<+>{Работает!}
    \only<+>{Ой, что-то сломалось. Зря я код не сохранил\dots}
\end{frame}

\begin{frame}[fragile]
    \frametitle{Архивы}
    \begin{block}{}
        \begin{minted}{bash}
$ ls versions
        \end{minted}
        \texttt{\textcolor[HTML]{aa0000}{v0.1.tar}} \quad
        \texttt{\textcolor[HTML]{aa0000}{v0.2.tar}} \quad
        \pause
        \texttt{\textcolor[HTML]{aa0000}{v0.3.tar}} \quad
        \texttt{\textcolor[HTML]{aa0000}{v0.4.tar}} \quad
        \texttt{\textcolor[HTML]{aa0000}{v0.5.tar}} \quad
        \texttt{\textcolor[HTML]{aa0000}{v0.6.tar}} \quad
        \texttt{\textcolor[HTML]{aa0000}{v0.7.tar}} \quad
        \texttt{\textcolor[HTML]{aa0000}{v0.8.tar}} \quad
        \texttt{\textcolor[HTML]{aa0000}{v0.9.tar}}
    \end{block}
    \begin{itemize}
        \pause
        \item[$+$] Просто
        \pause
        \item[$-$] Неудобно
    \end{itemize}
\end{frame}

\subsection{Резервировное копирование}
\begin{frame}[fragile]
    \frametitle{Несчастные случаи}
    \begin{itemize}
        \pause
        \item Украли ноутбук
        \pause
        \item Утопили ноутбук
        \pause
        \item Умер жёсткий диск
    \end{itemize}
\end{frame}

\subsection{Резервировное копирование}
\begin{frame}[fragile]
    \frametitle{Больше архивов}
    \begin{itemize}
        \item На флешке
        \item На Яндекс.Диске
    \end{itemize}
\end{frame}

\subsection{Командная работа}
\begin{frame}[fragile]
    \frametitle{Организация командной работы}
    Чат в телеграме:
    \begin{itemize}
        \pause
        \item[A:] Так, я сейчас сделаю это.
        \pause
        \item[B:] А я это сделаю.
        \pause
        \item[C:] Ой, тут баг есть, сейчас поправлю.
        \pause
        \item[A:] Стой! Не трогай тот код, я его сейчас изменяю.
        \pause
        \item[D:] Привет, я вернулся. У кого сейчас последняя версия исходников?
        \pause
        \item[B:] Я закончил.
        \pause
        \item[A:] Блин, я тоже тот файл правил...
    \end{itemize}
\end{frame}

\begin{frame}[fragile]
    \frametitle{Организация командной работы}
    Больше людей "--- ещё хуже.
    \pause
    Как решать проблему?
\end{frame}

\section{Принцип работы}

\section{Основные команды}

\section{Советы}

\section{Продвинутые команды}

\end{document}
